%% LyX 2.1.3 created this file.  For more info, see http://www.lyx.org/.
%% Do not edit unless you really know what you are doing.
\documentclass[french,english]{article}
\usepackage[T1]{fontenc}
\usepackage[latin9]{inputenc}
\usepackage{tipa}
\usepackage{babel}
\makeatletter
\addto\extrasfrench{%
   \providecommand{\og}{\leavevmode\flqq~}%
   \providecommand{\fg}{\ifdim\lastskip>\z@\unskip\fi~\frqq}%
}

\makeatother
\begin{document}
\selectlanguage{french}%
Coefficients de scattering de $x_{2}(u_{1},\lambda_{1},\lambda_{2})$
en fonction du temps $u_{1}$ et de la log-fr�quence $\log_{2}\lambda_{1}$,
pour $\lambda_{2}=(a,b,c)$ fix� avec $a^{-1}=120\,\mathrm{ms}$,
$b^{-1}=\pm1\,\mathrm{octave}$, $c^{-1}=\pm4\,\mathrm{octaves}$.
On constate que la syllabe /\textprimstress la\textsci / active en
particulier les coefficients tels que $b>0$, $c>0$ (hauteur montante,
timbre montant) tandis que /\textsci \textschwa n/ active les coefficients
tels que $b<0$, $c<0$ (hauteur descendante, timbre descendant).
La clart� est inversement proportionelle � l'amplitude des coefficients.\selectlanguage{english}%

\end{document}
