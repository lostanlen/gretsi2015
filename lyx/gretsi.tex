%% LyX 2.1.3 created this file.  For more info, see http://www.lyx.org/.
%% Do not edit unless you really know what you are doing.
\documentclass[french]{article}
\usepackage[T1]{fontenc}
\usepackage{babel}
\makeatletter
\addto\extrasfrench{%
   \providecommand{\og}{\leavevmode\flqq~}%
   \providecommand{\fg}{\ifdim\lastskip>\z@\unskip\fi~\frqq}%
}

\makeatother
\begin{document}

\title{Transform\'{e}e de scattering temps-chroma-octave}

\maketitle
Un d\'{e}fi majeur de la classification automatique de sons repose
sur une mod\'{e}lisation efficace de leur structure transitoire sur
des \'{e}chelles temporelles aussi longues que possible. De par sa
bonne localisation temps-fr\'{e}quence et sa facult\'{e} de r\'{e}gularisation
des signaux modul\'{e}s, un op\'{e}rateur non-lin\'{e}aire tel que
le module de la transform\'{e}e en ondelettes est un premier pas naturel
dans ce sens. Cependant, celui-ci est incapable de capturer, par simple
int\'{e}gration temporelle, des \'{e}l\'{e}ments acoustiques plus
riches tels que les variations de fr\'{e}quence fondamentale (\emph{chirps})
ou de profil formantique (coarticulations, attaques intrumentales).
Or, si le cas des \emph{chirps} et de la variabilit\'{e} harmonique
ont \'{e}t\'{e} abord\'{e}s ind\'{e}pendamment ({[}Gribonval{]}, {[}Peeters
et al.{]}), il n'existe pas d'approche syst\'{e}matique qui rende
compte de la dynamique jointe de ces deux facteurs.

Dans cette communication, nous introduisons une nouvelle repr\'{e}sentation
des sons, construite \`{a} partir du module de la transform\'{e}e
en ondelettes, visant explicitement \`{a} caract\'{e}riser les changements
de hauteur et de timbre. Dans une premi\`{e}re partie, nous montrons
comment enrouler l'axe fr\'{e}quentiel en une spirale des hauteurs
de sorte qu'un tour complet correspond \`{a} une transposition d'une
octave, afin de s\'{e}parer hauteur relative et registre global. Par
la suite, nous d\'{e}finissons un op\'{e}rateur unitaire et multi-\'{e}chelles
sur la spirale obtenue, construit comme une cascade de trois transform\'{e}es
en ondelettes \`{a} valeurs complexes.
\end{document}
